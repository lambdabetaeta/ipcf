We will use a variant of the proof in \cite{Kavvos2017b}, i.e.
the method of \emph{parallel reduction}. This kind of proof was
originally discovered by Tait and Martin-L\"of, and is nicely
documented in \cite{Takahashi1995}. Because of the intensional
nature of our $\ibox{(-)}$ constructs, ours will be more nuanced
and fiddly than any in \emph{op. cit.} The method is this: we will
introduce a second notion of reduction, \[
  \redp\ \subseteq \Lambda \times \Lambda
\] which we will `sandwich' between reduction proper and its
transitive closure: \[
  \red{}\ \subseteq\ \redp\ \subseteq\ \redt
\] We will then show that $\redp$ has the diamond property. By the
above inclusions, the transitive closure $\redp^\ast$ of $\redp$
is then equal to $\redt$, and hence $\red$ is Church-Rosser.

In fact, we will follow \cite{Takahashi1995} in doing something
better: we will define for each term $M$ its \emph{complete
development}, $M^\star$. The complete development is intuitively
defined by `unrolling' all the redexes of $M$ at once. We will
then show that if $M \redp N$, then $N \redp M^\star$. $M^\star$
will then suffice to close the diamond: \[
  \begin{tikzcd}
    & M
	\arrow[dr, Rightarrow]
	\arrow[dl, Rightarrow]
    &  \\
    P
      \arrow[dr, Rightarrow, dotted]
    & 
    & Q 
      \arrow[dl, Rightarrow, dotted] \\
    & M^\star
  \end{tikzcd}
\]

\begin{figure}
  \centering
  \caption{Parallel Reduction}
    \renewcommand{\arraystretch}{3}

\begin{center}
\begin{tabular}{c c}

  %% BETA

  $
    \begin{prooftree}
        \justifies
      M \redp{}M
        \using
      {(\textsf{refl})}
    \end{prooftree}
  $

  &

  $
    \begin{prooftree}
      M \redp{}N
        \quad\quad
      P \redp{}Q
        \justifies
      (\lambda x : A.\ M)P \redp{}N[Q/x]
        \using
      {(\rightarrow\beta)}
    \end{prooftree}
  $

  \\

  $
    \begin{prooftree}
      M \redp{}N
        \justifies
      \lambda x : A.\ M \redp{}\lambda x : A.\ N
        \using
      {(\textsf{cong}_\lambda)}
    \end{prooftree}
  $

  &

  $
    \begin{prooftree}
      M \redp{}N
        \quad\quad
      P \redp{}Q
        \justifies
      MP \redp{}NQ
        \using
      {(\textsf{app})}
    \end{prooftree}
  $

  \\

  $
    \begin{prooftree}
      P \redp{} P'
        \justifies
      \supset_G\ \textsf{true}\ P\ Q \redp{} P'
        \using
      {(\supset_1)}
    \end{prooftree}
  $

  & 

  $
    \begin{prooftree}
      Q \redp{} Q'
        \justifies
      \supset_G\ \textsf{false}\ P\ Q \redp{} Q'
        \using
      {(\supset_2)}
    \end{prooftree}
  $

  \\

  \multicolumn{2}{c}{
    $ \begin{prooftree}
      M \redp{} N
        \justifies
      \letbox{u}{\ibox{P}}{M} \redp{} N[P/u]
        \using
      {(\Box\beta)}
    \end{prooftree} $
  }

  \\

  \multicolumn{2}{c}{
    $
      \begin{prooftree}
	M \redp{} N
          \justifies
        \fixlob{z}{M} \redp{} N[\ibox{(\fixlob{z}{M})}/z]
          \using
        {(\Box\textsf{fix})}
      \end{prooftree}
    $
  }

  \\ 

  \multicolumn{2}{c}{
    $
      \begin{prooftree}
        \text{$M$ closed, $M \in \textrm{dom}(f)$}
          \justifies
        \tilde f(\ibox{M}) \redp{}\ibox{f(M)}	
          \using
        {(\Box\textsf{int})}
      \end{prooftree}
    $
  }

  \\

  \multicolumn{2}{c}{
    $
      \begin{prooftree}
        M \redp{}N
          \quad\quad
        P \redp{}Q
          \justifies
        \letbox{u}{M}{P} \redp{} \letbox{u}{N}{Q}
          \using
        {(\Box\textsf{let-cong})}
      \end{prooftree}
    $
  }

  \\

  \multicolumn{2}{c}{
    \begin{minipage}{\textwidth}
      \textbf{Remark}. In addition to the above, one should also
      include rules for the constants, but these are merely
      restatements of the rules in Figure \ref{fig:ipcfbeta}.
    \end{minipage}
  }

\end{tabular}
\end{center}


    \begin{minipage}{0.85\textwidth}
      \textbf{Remark}. In addition to the above, one should also include rules for the constants,
      but these are restatements of the rules in Figure \ref{fig:ipcfbeta}.
    \end{minipage}
  \label{fig:parallel}
\end{figure}

The parallel reduction $\redp$ is defined in Figure
\ref{fig:parallel}. Instead of the axiom $(\textsf{refl})$ we
would more commonly have an axiom for variables, $x \redp{} x$,
and $M \redp{} M$ would be derivable. However, we do not have a
congruence rule neither for $\ibox{(-)}$ nor for L\"ob's rule, so
that possibility would be precluded. We are thus forced to include
$M \redp{} M$, which slightly complicates the lemmas that follow.

The main lemma that usually underpins the confluence proof is
this: if $M \redp{} N$ and $P \redp{} Q$, $M[P/x] \redp{} N[Q/x]$.
However, this is intuitively wrong: no reductions should happen
under boxes, so this should only hold if we are substituting for a
variable \emph{not} occurring under boxes.  Hence, this lemma
splits into three different ones:
\begin{itemize}
  \item
    $P \redp{} Q$ implies $M[P/x] \redp{} M[Q/x]$, if $x$ does not
    occur under boxes: this is the price to pay for replacing the
    variable axiom with (\textsf{refl}).
  \item
    $M \redp{} N$ implies $M[P/u] \redp{} N[P/u]$, even if $u$ is
    under a box.
  \item
    If $x$ does not occur under boxes, $M \redp{} N$ and $P
    \redp{} Q$ indeed imply $M[P/x] \redp{} N[Q/x]$
\end{itemize} 
\begin{lem}
  \label{lem:substint}
  If $M \redp{} N$ then $M[P/u] \redp{} N[P/u]$.
\end{lem}
\begin{proof}
  By induction on the generation of $M \redp{} N$. Most cases
  trivially follow, or consist of simple invocations of the IH. In
  the case of $(\rightarrow\beta)$, the known substitution lemma
  suffices. Let us look at the cases involving boxes.
  \begin{indproof}
    \indcase{$\Box\beta$}
      Then $M \redp{} N$ is $\letbox{v}{\ibox{R}}{S} \redp{}
      S'[R/v]$ with $S \redp{} S'$. By the IH, we have that
      $S[P/u] \redp{} S'[P/u]$, so \[
	\letbox{v}{\ibox{R[P/u]}}{S[P/u]} \redp S'[P/u][R[P/u]/v]
      \] and this last is $\alpha$-equivalent to $S'[R/v][P/u]$ by
      the substitution lemma.
    \indcase{$\Box\textsf{fix}$}
      A similar application of the substitution lemma.
    \indcase{$\Box\textsf{int}$}
      Then $M \redp{} N$ is $\tilde f(\ibox{Q}) \redp{}
      \ibox{f(Q)}$. Hence \[
        \left(\tilde f(\ibox{Q})\right)[P/u]
          \equiv
        \tilde f(\ibox{Q})
          \redp{}
        \ibox{f(Q)}
          \equiv
        \left(\ibox{f(Q)}\right)[P/u]
    \] simply because both $Q$ and $f(Q)$ are closed.
  \end{indproof}
\end{proof}

\begin{lem} 
  \label{lem:substredp}
  If $P \redp Q$ and $x \not\in \bfv{M}$, then $M[P/x] \redp
  M[Q/x]$.
\end{lem}
\begin{proof}
  By induction on the term $M$. The only non-trivial cases are
  those for $M$ a variable, $\ibox{M'}$ or $\fixlob{z}{M'}$. In
  the first case, depending on which variable $M$ is, use either
  $(\textsf{refl})$, or the assumption $P \redp Q$. In the latter
  two, $(\ibox{M'})[P/x] \equiv \ibox{M'} \equiv (\ibox{M'})[Q/x]$
  as $x$ does not occur under a box, so use $(\textsf{refl})$, and
  similarly for $\fixlob{z}{M'}$.
\end{proof}

\begin{lem} 
  \label{lem:redp}
  If $M \redp N$, $P \redp Q$, and $x \not\in \bfv{M}$, then \[
    M[P/x] \redp N[Q/x]
  \]
\end{lem}

\begin{proof}
  By induction on the generation of $M \redp N$. The cases for
  most congruence rules and constants follow trivially, or from
  the IH. We prove the rest.
  \begin{indproof}
    \indcase{$\textsf{refl}$}
      Then $M \redp N$ is actually $M \redp M$, so we use Lemma
      \ref{lem:substredp} to infer $M[P/x] \redp M[Q/x]$. 

    \indcase{$\Box\textsf{int}$}
      Then $M \redp N$ is actually $\tilde f(\ibox{M}) \redp
      \ibox{f(M)}$. But $M$ and $f(M)$ are closed, so
      $\left(\tilde f(\ibox{M})\right)[P/x] \equiv \tilde
      f(\ibox{M}) \redp \ibox{f(M)} \equiv
      \left(\ibox{f(M)}\right)[Q/x]$.

    \indcase{$\supset_i$}
      Then $M \redp N$ is $\supset_G\ \textsf{true}\ M\ N \redp
      M'$ with $M \redp M'$. By the IH, $M[P/x] \redp M'[Q/x]$, so
      \[
	\supset_G\ \textsf{true}\ M[P/x]\ N[P/x]
	  \redp
	M'[Q/x]
      \] by a single use of $(\supset_1)$. The case for
      $\textsf{false}$ is similar.

    \indcase{$\rightarrow \beta$}
      Then $(\lambda x'{:}A.\ M)N \redp N'[M'/x']$, where $M \redp
      M'$ and $N \redp N'$. Then \[
	\left((\lambda x'{:}A.\ M)N\right)[P/x]
	  \equiv
	(\lambda x'{:}A.\ M[P/x])(N[P/x])
      \] But, by the IH, $M[P/x] \redp M'[Q/x]$ and $N[P/x] \redp
      N'[Q/x]$. So by $(\rightarrow \beta)$ we have \[
        (\lambda x'{:}A.\ M[P/x])(N[P/x])
          \redp
        M'[Q/x]\left[N'[Q/x]/x'\right]
      \] But this last is $\alpha$-equivalent to
      $\left(M'[N'/x']\right)\left[Q/x\right]$ by the substitution
      lemma.

    \indcase{$\Box\beta$}
      Then $\letbox{u'}{\ibox{M}}{N} \redp N'[M/u']$ where $N
      \redp N'$. By assumption, we have that $x \not\in \fv{M}$
      and $x \not\in \bfv{N}$. Hence, we have by the IH that
      $N[P/x] \redp N'[Q/x]$, so by applying $(\Box\beta)$ we get
      \begin{align*}
	(\letbox{u'}{\ibox{M}}{N})[P/x]
	  \equiv\
	    &\letbox{u'}{\ibox{M[P/x]}}{N[P/x]} \\
	  \equiv\
	    &\letbox{u'}{\ibox{M}}{N[P/x]} \\
	  \redp\
	    &N'[Q/x][M/u']
      \end{align*} But this last is $\alpha$-equivalent to
      $N'[M/u'][Q/x]$, by the substitution lemma and the fact that
      $x$ does not occur in $M$.

    \indcase{$\Box\textsf{fix}$}
      Then $\fixlob{z}{M} \redp{} M'[\ibox{(\fixlob{z}{M})}/z]$,
      with $M \redp{} M'$. As $x \not\in \bfv{\fixlob{z}{M}}$, we
      have that $x \not\in \fv{M}$, and by Lemma
      \ref{lem:varmon}, $x \not\in \fv{M'}$ either, so \[
        (\fixlob{z}{M})[P/x] \equiv \fixlob{z}{M}
      \] and\[
        M'[\fixlob{z}{M}/z][Q/x]
          \equiv
        M'[Q/x][\fixlob{z}{M[Q/x]}/z]
	  \equiv
	M'[\fixlob{z}{M}/z]
      \] Thus, a single use of $(\Box\textsf{fix})$ suffices.
  \end{indproof}
\end{proof}

We now pull the following definition out of the hat:
\begin{defn}[Complete development]
  The \emph{complete development} $M^\star$ of a term $M$ is
  defined by the following clauses:
    \begin{align*}
      x^\star
        &\defeq x \\
      c^\star
        &\defeq c \qquad (c \in 
	  \{\tilde f, \widehat{n},
	    \textsf{zero?}, \dots \}) \\
      \left(\lambda x{:}A.\ M\right)^\star
        &\defeq \lambda x{:}A.\ M^\star \\
      \left(\tilde{f}(\ibox{M})\right)^\star
	&\defeq \ibox{f(M)} \qquad \text{if $M$ is closed} \\
      \left(\left(\lambda x{:}A.\ M\right)N\right)^\star
        &\defeq M^\star[N^\star/x] \\
      \left(\supset_G\ \textsf{true}\ M\ N\right)^\star
        &\defeq M^\star \\
      \left(\supset_G\ \textsf{false}\ M\ N\right)^\star
        &\defeq N^\star \\
      \left(MN\right)^\star
        &\defeq M^\star N^\star \\
      \left(\ibox{M}\right)^\star
        &\defeq \ibox{M} \\
      \left(\letbox{u}{\ibox{M}}{N}\right)^\star
        &\defeq N^\star[M/u] \\
      \left(\letbox{u}{M}{N}\right)^\star
        &\defeq \letbox{u}{M^\star}{N^\star} \\
      \left(\fixlob{z}{M}\right)^\star
        &\defeq M^\star[\ibox{(\fixlob{z}{M})}/z]
    \end{align*}
\end{defn}

\noindent We need the following two technical results as well.

\begin{lem}
  \label{lem:reflstar}
  $M \redp{} M^\star$
\end{lem}
\begin{proof}
  By induction on the term $M$. Most cases follow immediately by
  (\textsf{refl}), or by the IH and an application of the relevant
  rule. The case for $\ibox{M}$ follows by $(\textsf{refl})$, the
  case for $\fixlob{z}{M}$ follows by $(\Box\textsf{fix})$, and
  the case for $\tilde f(\ibox{M})$ by $(\Box\textsf{int})$.
\end{proof}

\begin{lem}[BFV antimonotonicity]
  \label{lem:varmon}
  If $M \redp{} N$ then $\bfv{N} \subseteq \bfv{M}$.
\end{lem}
\begin{proof}
  By induction on $M \redp{} N$.
\end{proof}


\noindent And here is the main result:

\begin{thm}
  \label{thm:mainredp}
  If $M \redp P$, then $P \redp M^\star$.
\end{thm}

\begin{proof}
  By induction on the generation of $M \redp P$. The case of
  $(\textsf{refl})$ follows by Lemma \ref{lem:reflstar}, and the
  cases of congruence rules follow from the IH. We show the rest.
  \begin{indproof}
    \indcase{$\rightarrow\beta$}
      Then we have $(\lambda x{:}A.\ M)N \redp M'[N'/x]$, with $M
      \redp M'$ and $N \redp N'$. By the IH, $M' \redp M^\star$
      and $N' \redp N^\star$. We have that $x \not\in \bfv{M}$,
      so by Lemma \ref{lem:varmon} we get that $x \not\in
      \bfv{M'}$. Hence, by Lemma \ref{lem:redp} we get $M'[N'/x]
      \redp M^\star[N^\star/x] \equiv \left(\left(\lambda x{:}A.\
      M\right)N\right)^\star$.

    \indcase{$\Box\beta$}
      Then we have \[
	\letbox{u}{\ibox{M}}{N} \redp N'[M/u]
      \] where $N \redp N'$. By the IH, $N' \redp N^\star$, so it
      follows that \[
	N'[M/u] \redp N^\star[M/u] \equiv \left(
	\letbox{u}{\ibox{M}}{N}\right)^\star
      \] by Lemma \ref{lem:substint}.

    \indcase{$\Box\textsf{fix}$}
      Then we have \[
	\fixlob{z}{M}
	  \redp{}
	M'[\ibox{(\fixlob{z}{M})}/z]
      \] where $M \redp{} M'$. By the IH, $M' \redp{} M^\star$.
      Hence \[
	M'[\ibox{(\fixlob{z}{M})}/z]
	  \redp{}
	M^\star[\ibox{(\fixlob{z}{M})}/z]
	  \equiv
	\left(\fixlob{z}{M}\right)^\star
      \] by Lemma \ref{lem:substint}.
    \indcase{$\Box\textsf{int}$}
      Similar.
  \end{indproof}
\end{proof}

